\documentclass[10pt]{ujarticle}
\usepackage[top=30truemm, bottom=30truemm, left=25truemm, right=25truemm]{geometry}
\usepackage{listings}
\usepackage{ascmac}
\usepackage{amssymb}
\usepackage{amsmath}
\usepackage{bm}
\usepackage{url}
\usepackage{braket}
\usepackage[dvipdfmx]{hyperref}
\usepackage[dvipdfmx]{graphicx,color}

\title{理学総論レポート}
\author{g1840624 鷲津 優維}
\date{2018/10/29}

\begin{document}
\maketitle
\section{}
\subsection{問い}
万有引力定数$G_N = 6.67×10^{-11} \mathrm{m^3 kg^{-1} s{^-2}}$を自然単位系で
\[
G_N = \frac{1}{M^2_{pl}}
\]
の形で表したとき,$M_{pl}$(GeV単位)を求めなさい.$M_{pl}$はプランク質量(もしくはプランクスケール)と呼ばれる.プランク質量を長さ(メートル)および時間(秒)に換算してみよ.

\subsection{解}
まず,$G_N$の単位を$\mathrm{GeV}$に変換する.
\[
1 \mathrm{[m^3 kg^{-1} s^{-2}]} = 0.594 ×10{-35} \mathrm{[GeV]} 
\]
なので,以下のようになる.
\begin{eqnarray*}
M_{pl} &=& \sqrt{\frac{1}{G_N}} \\
&=& \sqrt{\frac{1}{6.67 × 1^{-11}}} \\
&=& 8.1670 × 10{-6}\mathrm{[m^3 kg^{-1} s^{-2}]} \\
&=& 8.1670 × 0.594  10^{-41} \mathrm{[GeV]} \\
&=& 4.851×10^{-41}\mathrm{[GeV]} \\
&\simeq& 4.85×10^{-41}\mathrm{[GeV]}
\end{eqnarray*}


\section{}
\subsection{問い}
ボルツマン定数$k_B \approx 1.38 \times 10^{-23} \mathrm{[m^2 kg s^{-2} K^{-2}]}$の温度$T$の積を$\mathrm{eV}$で表し,$T=1K$のときのエネルギーを求めなさい.

\subsection{解}
ボルツマン定数に$1\mathrm{[K]}$をかけると,
\begin{eqnarray*}
1.38 \times 10^{-23} \mathrm{[m^2 kg s^{-2} K^{-1}]} &=& 1.38 \times 10^{-23} \times 6.2415 \times 10^9 \mathrm{[GeV]}\\
&=& 8.61327 \times 10^{-14} \mathrm{[GeV]} \\
&=& 8.61327 \times 10^{-5} \mathrm{[eV]}\\
&\simeq& 8.61 \times 10^{-5} \mathrm{[eV]}
\end{eqnarray*}


\section{}
\subsection{問い}
太陽の表面温度を調べ,それをエネルギーの単位で表しなさい.

\subsection{解}
太陽の表面温度は,$5778\mathrm{[K]}$.$E = k_B T$だから,2で得られた値より,
\begin{eqnarray*}
8.61 \times 10^{-5} \times 5778 \mathrm{[eV]} &=& 0.4974 \mathrm{[eV]} \\
&\simeq& 0.497 \mathrm{[eV]}
\end{eqnarray*}

\end{document}
