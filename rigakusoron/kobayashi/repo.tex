\documentclass[10pt]{ujarticle}
\usepackage[top=30truemm, bottom=30truemm, left=25truemm, right=25truemm]{geometry}
\usepackage{listings}
\usepackage{ascmac}
\usepackage{amssymb}
\usepackage{amsmath}
\usepackage{bm}
\usepackage{url}
\usepackage{braket}
\usepackage[dvipdfmx]{hyperref}
\usepackage[dvipdfmx]{graphicx,color}

\title{理学総論レポート}
\author{g1840624 鷲津 優維}
\date{2018/11/26}

\begin{document}
\maketitle
\section{}
\subsection{問い}
スピン-軌道相互作用について

\subsection{解}
電子は質量,電荷の他に,スピン角運動量という粒子固有な物理量を持っている.スピン角運動量は磁場に応答する磁気モーメントの起源であり,電子は磁場中で2つの異なったエネルギー状態に分離している.一方,電荷をもつ電子が,軌道運動することにより,この軌道電流が磁場を生み出す.スピン軌道相互作用はDirac方程式から以下のように導出される.
Diracが提案した式(Dirac方程式)は以下のよう
\begin{eqnarray}
E
\left[
\begin{array}{c}
 \phi_1 \\
 \phi_2 \\
 \phi_3 \\
 \phi_4 
\end{array}
\right]
=
\left[
 \begin{array}{cccc}
   m_e c^2 & 0 & c\hat{p}_z & c(\hat{p}_x-i\hat{p}_y) \\
   0 & m_e c^2 & c(\hat{p}_x+i\hat{p}_y)&-c\hat{p}_z \\
   c\hat{p}_z & c(\hat{p}_x -i\hat{p}_y & -m_e c^2 & 0 \\
   c(\hat{p}_x + i\hat{p}_y) & -c\hat{p}_z & 0 & -m_e c^2 \\
 \end{array}
\right]
\left[
\begin{array}{c}
 \phi_1 \\
 \phi_2 \\
 \phi_3 \\
 \phi_4 \\
\end{array}
\right]
\end{eqnarray}

エネルギー演算子が$4\times4$行列であり,波動関数が4成分を持っている.4つの関数$\phi_1, \phi_2, \phi_3, \phi_4$がセットになって1つの電子状態が決まることがDirac方程式の特徴である.4成分の波動関数を$\bm{\phi}$とし,
\begin{eqnarray}
\bm{\phi}
=
\left[
\begin{array}{c}
 \phi_1 \\
 \phi_2 \\
 \phi_3 \\
 \phi_4 \\
\end{array}
\right]
\end{eqnarray}
エネルギー演算子の部分を$\hat{H}_D$と表記すると,
\begin{eqnarray}
\hat{H}_D \bm{\phi} = E \bm{\phi}
\end{eqnarray}

$\hat{H}_D$の右上の4成分はパウリのスピン行列で表現できる.
\begin{eqnarray}
\left[
\begin{array}{cc}
  c \hat{p}_z & c(\hat{p}_x-i\hat{p}_y) \\
  c(\hat{p}_x+i\hat{p}_z) & -c \hat{p}_z 
\end{array}
\right]
&=&
\left[
\begin{array}{cc}
  0 & 1 \\
  1 & 0
\end{array}
\right]
\hat{p}_x
+
\left[
\begin{array}{cc}
  0 & -i \\
  -i & 0
\end{array}
\right]
\hat{p}_y
+
\left[
\begin{array}{cc}
  1 & 0 \\
  0 & -1
\end{array}
\right]
\hat{p}_z \\
&=&
\bm{\sigma} \cdot \bm{\hat{p}}
\end{eqnarray}

次に,Dirac方程式を以下のように2成分にかき分ける
\begin{eqnarray}
\bm{\phi}=
\left[
\begin{array}{c}
  \phi_L \\
  \phi_S \\
\end{array}
\right]
\end{eqnarray}
$\phi_L$と$\phi_S$はそれぞれ右巻き,左巻きのスピンを表している.これらを用いてDirac方程式は
\begin{eqnarray}
E
\left[
\begin{array}{c}
  \phi_L \\
  \phi_S 
\end{array}
\right]
=
m_e c^2
\left[
\begin{array}{cc}
  \bm{I_2} & 0_2 \\
  0_2 & -\bm{I}_2
\end{array}
\right]
\left[
\begin{array}{c}
  \phi_L \\
  \phi_S 
\end{array}
\right]
+
\left[
\begin{array}{cc}
  0_2 & c \bm{\sigma}\cdot \bm{\hat{p}} \\
  c \bm{\sigma}\cdot \bm{\hat{p}} & 0_2
\end{array}
\right]
\left[
\begin{array}{c}
  \phi_L \\
  \phi_S 
\end{array}
\right]
=
\left[
\begin{array}{cc}
  m_e c^2 I & c \bm{\sigma}\cdot \bm{\hat{p}} \\
  c \bm{\sigma}\cdot \bm{\hat{p}} & -m_e c^2 I
\end{array}
\right]
\left[
\begin{array}{c}
  \phi_L \\
  \phi_S 
\end{array}
\right]
\end{eqnarray}

と表せる.

エネルギー基準を定義し直して,$E-m_e c^2$を新たな$E$とする.
\begin{eqnarray}
E
\left[
\begin{array}{c}
  \phi_L \\
  \phi_S
\end{array}
\right]
=
\left[
\begin{array}{cc}
  0_2 & c \bm{\sigma} \cdot \bm{\hat{p}} \\
  c \bm{\sigma} \cdot \bm{\hat{p}} & -2m_e c-2 \bm{I}_2 
\end{array}
\right]
\left[
\begin{array}{c}
  \phi_L \\
  \phi_S
\end{array}
\right]
\end{eqnarray}
さらにポテンシャルエネルギー$V$を含めて考えると,運動エネルギーは$E-V$となるので,行列を分解して書くと,
\begin{eqnarray}
\label{eq1}
  E \phi_L &=& V \phi_L + c\bm{\sigma} \cdot \bm{\hat{p}} \phi_S \\
\label{eq2}
  E \phi_S &=& c\bm{\sigma} \cdot \bm{\hat{p}} \phi_L + (V-2m_e c^2) \phi_S 
\end{eqnarray}

(\ref{eq2})より,
\begin{eqnarray}
  \phi_S = \frac{1}{E-V+2m_e c^2} (c \bm{\sigma} \cdot \bm{\hat{p}}) \phi_L
\end{eqnarray}
これを(\ref{eq1})に代入する.
\begin{eqnarray}
  \label{eq5}
  E \phi_L = V\phi_L + (c \bm{\sigma} \cdot \bm{\hat{p}} \frac{1}{E-V+2m_e c^2} (c\bm{\sigma} \cdot \bm{\hat{p}}) \phi_L
\end{eqnarray}

となる.ここで,$E-V<<2m_e c^2$つまり,$E-V+2m_e c^2 \simeq 2m_e c^2$とすると,
\begin{eqnarray}
  \phi_S &\simeq& \frac{1}{2m_e c^2} (c \bm{\sigma} \cdot \bm{\hat{p}}) \phi_L = \frac{\bm{\sigma} \cdot \bm{\hat{p}}}{2m_e c^2} \phi_L \\
  E \phi_L &=& V \phi_L + (c \bm{\sigma} \cdot \bm{\hat{p}})\frac{1}{E-V+2m_e c^2} (c \bm{\sigma} \cdot \bm{\hat{p}}) \phi_L \nonumber \\
  &\simeq& V \phi_L + (c \bm{\sigma} \cdot \bm{\hat{p}})\frac{1}{2m_e c^2} (c \bm{\sigma} \cdot \bm{\hat{p}}) \phi_L \nonumber \\
  \label{eq3}
  &=& \left( \frac{\bm{\hat{p}}}{2m_e} + V \right) \phi_L
\end{eqnarray}

(\ref{eq3})をについて
\begin{eqnarray}
  E\phi_1 &=& \left( \frac{\bm{\hat{p}}^2}{2m_e} + V \right) \phi_1 \\
  E\phi_2 &=& \left( \frac{\bm{\hat{p}}^2}{2m_e} + V \right) \phi_2
\end{eqnarray}

このように,相対論効果が小さいとDirac方程式の$\phi_L$はシュレディンガー方程式の解と同じになる.波動関数を規格化すると,
\begin{eqnarray*}
  \int \bm{\phi}^* \bm{\phi} dv = \int \phi_L^* \phi_L dv + \int \phi_S^* \phi_S dv = 1
\end{eqnarray*}

近似を用いて簡単に考えると,
\begin{eqnarray}
\label{eq4}
  \int \phi_L^* \phi_L dv + \int \phi_S^* \phi_S dv &=& \int \phi_L^* \phi_L dv + \int \left( \frac{\bm{\hat{p}}^2}{2m_e} \phi_L \right) ^*\frac{\bm{\hat{p}}^2}{2m_e} \phi_L dv \nonumber \\
  &=& \int \phi_L^* \left[ 1+ \frac{\bm{\hat{p}}^2}{4m_e^2 c^2} \right] \phi_L dv \nonumber \\
  &=& \int \phi_L^* \left[ 1+ \frac{\bm{\hat{p}}^2}{8m_e^2 c^2} \right] \left[ 1+ \frac{\bm{\hat{p}}^2}{8m_e^2 c^2} \right] \phi_L dv \nonumber \\
  &=& 1
\end{eqnarray}

(\ref{eq4})の式は$\phi_S$の効果も取り込んで規格化された2成分の波動関数として機能する波動関数($\phi_T$とする.)の形を示している.

\begin{eqnarray}
  \label{eq6}
  \phi_T = \left[ 1+ \frac{\bm{\hat{p}}^2}{8m_e^2 c^2} \right] \phi_L, \phi_L=\left[ 1- \frac{\bm{\hat{p}}^2}{8m_e^2 c^2} \right] \phi_L
\end{eqnarray}

(\ref{eq5})を,(\ref{eq6})を考慮して$\phi_T$の方程式に書き換えると,
\begin{eqnarray}
\left[ 1- \frac{\bm{\hat{p}}^2}{8m_e^2 c^2} \right] \left[(c \bm{\sigma} \cdot \bm{\hat{p}} ) \frac{1}{E-V+2m_e c^2} (c\bm{\sigma} \cdot \bm{\hat{p}})+V \right] \left[ 1- \frac{\bm{\hat{p}}^2}{8m_e^2 c^2} \right] \phi_T = E \left[ 1- \frac{\bm{\hat{p}}^2}{8m_e^2 c^2} \right] ^2 \phi_T
\end{eqnarray}

$E-V<<2m_e c^2$を考慮すると,
\begin{eqnarray}
  \left| \frac{E-V}{2m_e c^2} \right| << 1 
\end{eqnarray}
ここからテイラー展開を利用して,近似すると,
\begin{eqnarray}
  \frac{c^2}{E-V+2m_e c^2} = \frac{c^2}{2m_e c^2 + (E-V)} = \frac{1}{2m_e} \frac{1}{1+\frac{E-V}{2m_e c^2}} \simeq \frac{1}{2m_e} \left[ 1- \frac{E-V}{2m_e c^2} \right]
\end{eqnarray}
となるので,

\begin{eqnarray}
\left[ 1- \frac{\bm{\hat{p}}^2}{8m_e^2 c^2} \right] \left[(\bm{\sigma} \cdot \bm{\hat{p}} ) \frac{1}{2m_e} \left[ 1- \frac{E-V}{2m_e c^2} \right] (\bm{\sigma} \cdot \bm{\hat{p}})+V \right] \left[ 1- \frac{\bm{\hat{p}}^2}{8m_e^2 c^2} \right] \phi_T = E \left[ 1+ \frac{\bm{\hat{p}}^2}{4m_e^2 c^2} \right] ^2 \phi_T
\end{eqnarray}

これより,
\begin{eqnarray}
  \label{eq7}
  \left[ \frac{\bm{\hat{p}}^2}{2m}+V -\frac{\bm{\hat{p}}^4}{8m_e^3 c^2} - \frac{\hbar \bm{\sigma} \cdot(\nabla \cdot V \times \bm{\hat{p}})}{4m_e^2 c^2} + \frac{\hbar^2}{8m_e^2 c^2} \delta V \right] \phi_T = E\phi_T
\end{eqnarray}

原子を仮定すると,$V=\frac{Ze^2}{4\pi \epsilon_0 r}$なので,(\ref{eq7})の第4項がスピン軌道相互作用項となる.
\begin{eqnarray}
  \hat{H}_{\rm{spin-orbit}} &=& -\frac{1}{4m_e^2 c^2} i\sigma \left[\bm{\hat{p}} V\right] \times \bm{\hat{p}} = \frac{1}{4m_e^2 c^2} i\sigma \left[\bm{\hat{p}} \frac{e^2}{4\pi \epsilon_0 r}\right] \times \bm{\hat{p}} = \frac{1}{4m_e^2 c^2} i\sigma \left[ \frac{\hbar}{i} \nabla \cdot \frac{e^2}{4r\pi \epsilon_0} \right] \times \bm{\hat{p}} \\
  &=& \frac{1}{4 m_e^2 c^2}\hbar \bm{\sigma} \frac{e^2}{4\pi \epsilon_0} \left[ \nabla \cdot \frac{1}{r} \right] \times \bm{\hat{p}} = \frac{1}{4m_e^2 c^2} \frac{e^2}{4\pi \epsilon_0} \hbar \bm{\sigma} \left[ \frac{\bm{r}}{r^3} \right] \times \bm{\hat{p}} \\
  &=& \frac{1}{2m_e^2 c^2} \frac{Ze^2}{4\pi \epsilon_0} \left[ \frac{1}{2} \bm{\sigma} \hbar \right] \frac{1}{r^3} \bm{\hat{L}}
\end{eqnarray}

ここで,$\bm{\hat{L}}$はスピン角運動量を表す.\\
このように,Dirac方程式からスピン軌道相互作用を導出することができた.







%
%$\hat{H}_D$を2乗すると,
%\begin{eqnarray}
%\hat{H}_D^2
%=
%\left[
%\begin{array}{cccc}
%  m_e^2 c^4 + c^2 \bm{p}^2 & 0 & 0 & 0 \\
%  0 & m_e^2 c^4 + c^2 \bm{p}^2 & 0 & 0 \\
%  0 & 0 & m_e^2 c^4 + c^2 \bm{p}^2 & 0 \\
%  0 & 0 & 0 & m_e^2 c^4 + c^2 \bm{p}^2 
%\end{array}
%\right]
%

\end{document}
