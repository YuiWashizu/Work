\documentclass[10pt]{ujarticle}
\usepackage[top=30truemm, bottom=30truemm, left=25truemm, right=25truemm]{geometry}
\usepackage{listings}
\usepackage{ascmac}
\usepackage{amssymb}
\usepackage{amsmath}
\usepackage{bm}
\usepackage{url}
\usepackage{braket}
\usepackage[dvipdfmx]{hyperref}
\usepackage[dvipdfmx]{graphicx,color}

\title{量子物理学特論レポート}
\author{g1840624 鷲津 優維}
\date{2018/11/14}

\begin{document}
\maketitle
\section{問い}
行列表示にて,以下のように$\rho$が表される時,
\[
\rho_s(t) = \left(
\begin{array}{cc}
|\alpha|^2 & \alpha \beta^{\ast} e^{-i\omega_0 t} C(t) \\
\alpha^{\ast} \beta e^{i\omega_0 t} C(t) & |\beta|^2
\end{array}
\right)
\]
スピンの期待値は
\begin{eqnarray*}
\langle S_\pm \rangle_t &=& \langle S_x \rangle_t \pm i \langle S_y \rangle_t \\
\langle S_z \rangle_t &=& \mathrm{Tr}_S S_z \rho_x(t) \\
&=& \frac{1}{2} (|\alpha|^2 - |\beta|^2)\\
\end{eqnarray*}
と書け,このとき一般に
\[
\rho_s(t) = \left(
\begin{array}{cc}
\frac{1}{2} + \langle S_z \rangle_t & \langle S_- \rangle_t \\
\langle S_+ \rangle & \frac{1}{2} - \langle S_z \rangle_t
\end{array}
\right)
\]
と表せる.\\
このとき,エントロピー$S(t) = -k_B \mathrm{Tr} \rho_s (t) \ln \rho_s(t)$を用いて緩和過程の評価方法を検討せよ.

\section{解}
$\rho_s(t)$の式をエントロピー$S(t)$に代入すると,



\end{document}
